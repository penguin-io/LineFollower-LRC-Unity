\documentclass[12pt]{article}
\usepackage[margin=1in]{geometry}
\usepackage{setspace}
\usepackage{graphicx}
\usepackage{hyperref}
\usepackage{booktabs}
\usepackage{array}
\usepackage{tabularx}
\usepackage{longtable}
\usepackage{enumitem}
\usepackage{titlesec}
\usepackage{lipsum}
\setstretch{1.15}
\setlist[itemize]{topsep=4pt,itemsep=2pt,leftmargin=1.2cm}
\setlist[enumerate]{topsep=4pt,itemsep=2pt,leftmargin=1.4cm}
\newcolumntype{Y}{>{\raggedright\arraybackslash}X}
\hypersetup{
    colorlinks=true,
    linkcolor=blue,
    urlcolor=blue,
    citecolor=blue
}

\begin{document}

\begin{titlepage}
    \centering
    {\Large Lam Research Challenge 2025\\[0.5em]Logical League (Round 1)}\\[2em]
    {\Huge \bfseries Advanced Line Follower + Single Arm Robot (SARM)}\\[3em]
    {\Large Design Documentation}\par
    \vfill
    {\large Prepared by}\\[0.5em]
    {\Large \bfseries RoboPenguins}\\[0.25em]
    {National Institute of Technology Calicut}\\[2em]
    \begin{tabular}{ll}
        Team Leader: & Thiruchalvan Thiyagarajan (thiruchalvanthiagi@gmail.com)\\
        Member: & Harshan J (harshan\_b230968ee@nitc.ac.in)\\
        Member: & Mohammed Rehan Tadpatri (mohammed\_b231090pe@nitc.ac.in)\\
        Member: & Thareesh Prabakaran (developer.thareesh@gmail.com)
    \end{tabular}\\[3em]
    {\large Date: October 21, 2025}\\[4em]
    \vfill
    {\small Version 2.0}
\end{titlepage}

\pagenumbering{roman}
\tableofcontents
\clearpage
\pagenumbering{arabic}

\section{Team Profile}
\begin{itemize}
    \item \textbf{Team Name:} RoboPenguins
    \item \textbf{Institution:} National Institute of Technology Calicut
    \item \textbf{Members:}
    \begin{itemize}
        \item Thiruchalvan Thiyagarajan --- Team Leader --- \href{mailto:thiruchalvanthiagi@gmail.com}{thiruchalvanthiagi@gmail.com}
        \item Harshan J --- \href{mailto:harshan_b230968ee@nitc.ac.in}{harshan\_b230968ee@nitc.ac.in}
        \item Mohammed Rehan Tadpatri --- \href{mailto:mohammed_b231090pe@nitc.ac.in}{mohammed\_b231090pe@nitc.ac.in}
        \item Thareesh Prabakaran --- \href{mailto:developer.thareesh@gmail.com}{developer.thareesh@gmail.com}
    \end{itemize}
\end{itemize}

\section{Executive Summary}
This document captures our complete technical design for the Lam Research Challenge 2025, Logical League (Round 1). It satisfies the rulebook requirements by detailing:
\begin{itemize}
    \item \textbf{S.No 1} --- Arena circuitry design integrating pump, LED ``LAM'' display, load cell, and LCD.
    \item \textbf{S.No 2} --- Single Arm Robot (SARM) on an omnidirectional platform.
    \item \textbf{S.No 3} --- Advanced Line Follower Robot (ALFR).
    \item \textbf{S.No 7} --- Peristaltic pump for the fluid delivery station.
    \item \textbf{S.No 8} --- Custom SARM mechanical design and enhancements.
    \item Cross-cutting software architecture: ROS~2 middleware with Unity-based simulation for validation and iteration.
\end{itemize}
The document is organized for quick scoring alignment, with each numbered requirement addressed in targeted sections. Team-specific branding and calibration media will be inserted prior to submission.

\section{Competition Alignment Overview}
Two cooperative robots (autonomous ALFR and manual SARM) must clear obstacles, trigger three gate-based processes, and verify fluid delivery accuracy within a 10-minute time limit. Scoring prioritizes arena circuitry, robot capability, pump performance, and unique documentation.

\subsection{Technologies Selected}
\begin{itemize}
    \item \textbf{ROS 2 (Humble compatible):} primary middleware for modular control, inter-robot communication, and hardware abstraction.
    \item \textbf{Unity 2022 LTS with Robotics Hub packages:} high-fidelity simulation environment used to prototype kinematics, sensing, and gate flow logic before hardware deployment. Unity consumes URDF assets and ROS~2 topics via ROS-TCP-Connector.
\end{itemize}

\subsection{Key Objectives}
\begin{enumerate}
    \item Deliver a cohesive system architecture where ALFR and SARM cooperate without deadlocks at junctions.
    \item Guarantee the 125~ml fluid dispensing sequence and associated arena logic meet tolerance and safety expectations.
    \item Produce a distinct document reflecting team identity, custom mechanical work, and ROS~2/Unity integration methodology.
\end{enumerate}

\section{System Architecture Snapshot}
\begin{tabularx}{\textwidth}{>{\bfseries}lYYY}
\toprule
Element & Role & Key Technologies & Rulebook Mapping \\
\midrule
ALFR & Autonomous navigation, gate sequencing, line tracking & ROS~2 navigation stack (custom), Unity physics validation & S.No 3 \\
SARM & Manual obstacle removal, precision placement & ROS~2 teleop, custom CAD-designed omni base & S.No 2 \& S.No 8 \\
Arena Circuitry & Sensors and actuation for gates, pump, LED, load cell, LCD & Raspberry Pi GPIO, HX711, WS2812B LEDs & S.No 1 \\
Peristaltic Pump & 125 ml dispensing triggered by Gate 1 & Stepper + A4988/TMC2208, fluid calibration & S.No 7 \\
Software Bridge & Unity simulation, ROS~2 communications, data logging & ROS-TCP-Connector, ROS~2 nodes & All S.Nos \\
\bottomrule
\end{tabularx}

\subsection{Operational Flow}
\begin{enumerate}
    \item ALFR follows the arena line and halts at obstacles or blocked junction markers.
    \item SARM teleoperates to remove obstacles (three staged tasks), signaling clearance.
    \item Gate~1 crossing triggers fluid delivery; Gate~2 illuminates the LAM insignia; Gate~3 verifies dispense via load cell and updates the LCD.
    \item Unity simulation replicates this workflow, enabling parameter tuning prior to hardware trials.
\end{enumerate}

\section{Advanced Line Follower Robot (S.No 3)}\label{sec:alfr}
\subsection{Mechanical and Sensor Layout}
\begin{itemize}
    \item \textbf{Chassis:} low-profile acrylic or aluminum base with differential drive wheels, caster support, and centralized electronics bay.
    \item \textbf{Sensor array:} eight-channel reflective IR array positioned 18~mm ahead of the front axle, adjustable for line contrast. Each channel includes analog capture for centroid calculation and binary thresholding for junction detection.
    \item \textbf{Additional sensing:} forward-facing ultrasonic sensor (15$^{\circ}$ field of view) for obstacle anticipation and speed modulation near junctions.
    \item \textbf{Electronics:} STM32 or Teensy microcontroller interfaced with ROS~2 via micro-ROS or a secondary single-board computer; LiPo 3S battery with regulator rails (5~V control, 7.4~V motors).
\end{itemize}

\subsection{Control Strategy}
\begin{itemize}
    \item \textbf{Line tracking:} weighted center calculation with PD compensation. Gains are tuned through Unity simulation sweeps, then validated on hardware. A saturation block limits angular velocity to prevent overshoot on tight turns.
    \item \textbf{Junction policy:} multi-sensor activation triggers a state change to ``junction pending.'' If an obstacle flag remains true (received from SARM or gate sensors), ALFR enters ``hold'' mode until clearance notification arrives.
    \item \textbf{Gate interaction:} ROS~2 topic \texttt{/gates/gX\_crossed} is published when Unity or hardware sensors confirm crossing. This ensures deterministic activation of pump (Gate~1), LED array (Gate~2), and load cell check (Gate~3).
\end{itemize}

\subsection{Simulation-to-Hardware Transfer}
Unity replicates wheelbase, mass distribution, and IR sensing noise to approximate real-world behavior. ROS~2 nodes in simulation share the same interface as hardware nodes, reducing integration risk when transitioning to the physical arena.

\section{Single Arm Robot on Omni Platform (S.No 2 and S.No 8)}\label{sec:sarm}
\subsection{Custom Mechanical Design}
\begin{itemize}
    \item \textbf{Chassis frame:} bespoke hexagonal platform providing symmetric mounting points for four mecanum wheels. Laser-cut aluminum with cross-bracing maintains stiffness while keeping weight under 4~kg.
    \item \textbf{Arm assembly:} 4~DoF articulated arm (base rotation, shoulder, elbow, gripper pitch) plus a two-finger parallel gripper. Links are CNC-machined aluminum; joints use NEMA17 steppers with harmonic gearboxes for backlash reduction.
    \item \textbf{Electronics bay:} layered deck containing motor drivers, power distribution board, and on-board compute (Jetson Nano or Raspberry Pi~5) for ROS~2 nodes.
    \item \textbf{Cable management:} custom 3D-printed ducts route conductors to avoid interference with omnidirectional motion.
\end{itemize}

\subsection{Mobility and Control}
\begin{itemize}
    \item \textbf{Drive system:} mecanum wheels permit translation in any direction, enabling precise alignment alongside obstacles.
    \item \textbf{Teleoperation:} ROS~2 \texttt{joy\_node} feeds into \texttt{teleop\_twist\_joy}, producing \texttt{/sarm/cmd\_vel} Twist messages. Deadman switches ensure safety.
    \item \textbf{Arm kinematics:} ROS~2 \texttt{joint\_state\_broadcaster} and \texttt{joint\_trajectory\_controller} (or custom node) manage position commands. In simulation, Unity \texttt{ArticulationBody} joints mirror the hardware kinematics.
    \item \textbf{Obstacle workflow:}
    \begin{enumerate}
        \item Navigate to obstacle using orthogonal strafing.
        \item Deploy arm to grasp obstacle, using wrist pitch to lift.
        \item Place obstacle outside ALFR path and signal clearance via \texttt{/sarm/clearance}.
    \end{enumerate}
\end{itemize}

\subsection{Customization Highlights}
\begin{itemize}
    \item \textbf{Grip adaptability:} interchangeable fingertip pads (TPU inserts) accommodate cylindrical or rectangular obstacles.
    \item \textbf{Vision augmentation (optional):} forward camera feed displayed to the operator improves manual placement accuracy.
\end{itemize}

\section{Peristaltic Pump System (S.No 7)}\label{sec:pump}
\subsection{Hardware Architecture}
\begin{itemize}
    \item \textbf{Pump head:} tri-roller design with silicone tubing (3~mm inner diameter) secured by quick-release clamps for maintenance.
    \item \textbf{Actuation:} NEMA17 stepper motor coupled via 3D-printed flex coupler; microstepping through TMC2208 driver ensures smooth flow.
    \item \textbf{Control electronics:} Raspberry Pi GPIO pins control STEP/DIR. A 220~\textmu F bulk capacitor across VMOT mitigates transients, supplemented by flyback diode networks.
\end{itemize}

\subsection{Calibration and Accuracy}
\begin{itemize}
    \item \textbf{Steps-per-milliliter calibration:} empirical calibration using load cell data. Sample trial: 1000 microsteps correspond to approximately 15.2~ml; final conversion factor computed as 82~steps/ml for the 125~ml target. The factor is stored as a ROS~2 parameter for quick adjustments.
    \item \textbf{Closed-loop verification:} load cell readings at Gate~3 confirm delivery. If deviation exceeds \pm5~g, the system prompts manual inspection.
\end{itemize}

\subsection{Safety and Maintenance}
\begin{itemize}
    \item Tubing replaced every 20 cycles to preserve volumetric consistency.
    \item Pump assembly enclosed to prevent pinch hazards; a transparent window allows visual inspection.
    \item Emergency stop input cuts motor power via relay while preserving control electronics.
\end{itemize}

\section{Arena Circuitry Design (S.No 1)}\label{sec:arena}
\subsection{Functional Blocks}
\begin{enumerate}
    \item \textbf{Gate sensors:} three IR break-beam assemblies aligned with track positions. Outputs connect to Raspberry Pi GPIO with hardware debouncing.
    \item \textbf{Pump control:} STEP/DIR outputs drive an A4988 driver; a ``Pump ON'' indicator LED provides visual feedback.
    \item \textbf{LED ``LAM'' display:} 30-node WS2812B strip shaped to form the letters L-A-M, controlled via a Pi PWM-capable pin through a level shifter.
    \item \textbf{Load cell and HX711:} load measurement for the final stage; the board communicates via HX711 digital interface to the Pi.
    \item \textbf{LCD display (16\texttimes 2 I2C):} final messaging (team name, success/error prompts).
\end{enumerate}

\subsection{Power Distribution}
\begin{itemize}
    \item 12~V, 5~A main supply feeds pump motor, stepper driver, and downconverter.
    \item Buck converter provides regulated 5~V to LEDs, HX711, LCD, and sensors.
    \item Logic rails isolated via optocouplers for the pump driver to mitigate back electromotive force.
    \item Each peripheral includes inline fuses (1~A for logic, 3~A for motor) and transient voltage suppression diodes on sensor lines.
\end{itemize}

\subsection{Signal Topology}
\begin{itemize}
    \item All sensors share the Raspberry Pi ground reference. Shielded cables minimize load cell noise.
    \item Gate sensor signals filtered with RC networks (4.7~k\ensuremath{\Omega} and 0.1~\ensuremath{\mu}F) to suppress ambient light flicker.
    \item LCD I2C lines incorporate 4.7~k\ensuremath{\Omega} pull-ups; cable length kept under 30~cm.
\end{itemize}

\subsection{Enclosure and Access}
A modular arena control box houses the Pi, drivers, and wiring harness. A hinged door with interlock ensures power is disconnected before maintenance. Quick-connect headers simplify arena setup breakdown for transport.

\section{Software and Integration Stack}
\subsection{ROS 2 Architecture}
\begin{itemize}
    \item \textbf{Python nodes (located in \texttt{ROS2/lrc\_arena\_nodes/lrc\_arena\_nodes/}):}
    \begin{itemize}
        \item \texttt{pump\_controller}: accepts Gate~1 trigger and drives peristaltic pump STEP/DIR lines.
        \item \texttt{led\_controller}: listens for Gate~2 trigger and manages the WS2812B ``LAM'' array.
        \item \texttt{loadcell\_lcd\_controller}: subscribes to the Gate~3 trigger, samples the HX711 load cell, and updates LCD content.
        \item Optional extensions include teleoperation multiplexing and an ALFR watchdog for future iterations.
    \end{itemize}
    \item \textbf{Unity C\# scripts (located in \texttt{Assets/Scripts/}):}
    \begin{itemize}
        \item \texttt{IRArrayPublisher}: streams IR sensor data using the custom \texttt{IRArray} message.
        \item \texttt{SonarPublisher}: reports ultrasonic range for obstacle anticipation.
        \item \texttt{GateTriggerPublisher}: announces gate crossings (1--3) via Boolean topics.
        \item \texttt{CmdVelSubscriber}: converts \texttt{/sarm/cmd\_vel} Twist commands into mecanum wheel velocities.
        \item \texttt{GripperSubscriber}: maps normalized float commands to the SARM gripper articulation.
    \end{itemize}
    \item \textbf{Topics and services:}
    \begin{itemize}
        \item \texttt{/alfr/ir\_array} (\texttt{lrc\_arena\_msgs/IRArray}), \texttt{/alfr/sonar} (\texttt{sensor\_msgs/Range}), \texttt{/alfr/state} (\texttt{std\_msgs/String}).
        \item \texttt{/sarm/cmd\_vel} (\texttt{geometry\_msgs/Twist}), \texttt{/sarm/gripper/command} (\texttt{std\_msgs/Float32}).
        \item Gate triggers: \texttt{/gates/g1\_crossed}, \texttt{/gates/g2\_crossed}, \texttt{/gates/g3\_crossed}.
        \item Arena device feedback: \texttt{/pump/state}, \texttt{/pump/progress}, \texttt{/load\_cell/weight}, \texttt{/lcd/display}, \texttt{/led/lam\_state}.
        \item Optional services allow recalibration of pump throughput or updating LCD media from an operator console.
    \end{itemize}
\end{itemize}

\subsection{Unity Simulation Workflow}
\begin{itemize}
    \item URDF models of ALFR and SARM are prepared in SolidWorks, exported to URDF using Unity Robotics Hub guidelines, and imported through the URDF Importer package. Adjust articulation damping and ensure mesh scaling equals 1.0.
    \item The Unity scene (\texttt{Assets/Scenes/Main.unity}) embeds the arena mesh derived from provided CAD, gate colliders with trigger scripts, physics materials for line versus floor friction, and sensor emulation behaviours tied to the IR array and sonar publishers.
    \item ROS communication is bridged by ROS-TCP-Connector configured in ROS~2 protocol mode. Menu path: \texttt{Robotics \textrightarrow{} ROS Settings}. Default port is 10000. Message generation follows ROS--Unity Integration tutorials ensuring consistent namespace mapping.
    \item Pick-and-place tutorial patterns inspired the articulation hierarchy for the SARM arm and the usage of trajectories; wheel drive targets replace trajectory control when simulating mecanum locomotion.
    \item Simulation sessions emit CSV or JSON logs capturing line error, motor commands, and gate timestamps for later comparison with hardware bag files.
\end{itemize}

\subsection{Message and Package Management}
\begin{itemize}
    \item ROS~2 message package \texttt{lrc\_arena\_msgs} defines \texttt{IRArray.msg} and builds via \texttt{colcon}. Unity generates corresponding C\# message classes through \texttt{Robotics \textrightarrow{} Generate ROS Messages}.
    \item Arena node package \texttt{lrc\_arena\_nodes} uses \texttt{ament\_python}. Launch file \texttt{arena\_devices.launch.py} starts pump, LED, and verification nodes with configurable parameters such as \texttt{pump\_steps\_per\_ml} and \texttt{led\_count}.
    \item \texttt{rosdep install} resolves dependencies including \texttt{RPi.GPIO}, \texttt{hx711}, \texttt{rpi\_ws281x}, and \texttt{RPLCD}. Simulation-only runs use mock drivers that log to console.
\end{itemize}

\subsection{Tutorial Alignment and Enhancements}
Documentation cross-references ensure anyone familiar with Unity Robotics Hub tutorials can map this implementation back to canonical steps, simplifying onboarding for new team members.

\subsection{Validation Strategy}
\begin{itemize}
    \item \textbf{Digital twin alignment:} parameters (mass, friction, sensor offsets) iteratively tuned until Unity trajectories match hardware within \pm5\% for key maneuvers.
    \item \textbf{Stress tests:} scenario library covers worst-case obstacle placements, pump misfires, and communication delays. ROS~2 bag files capture outcomes for retrospective analysis.
\end{itemize}

\section{Custom Components and Identity Elements}\label{sec:custom}
\begin{itemize}
    \item \textbf{SARM chassis and arm:} fully bespoke CAD exported as neutral STL or STEP for manufacturing. Unity assets maintain accurate collision volumes.
    \item \textbf{Arena aesthetic:} team-branded LED patterns and LCD messages. The LED banner cycles ``RoboPenguins'' with a cyan-to-white gradient, and LCD splash screens display the team roster during idle states.
    \item \textbf{Documentation identity:} this design document references team-specific figures and parameter choices, meeting S.No 9 scoring for distinctive documentation.
\end{itemize}

\section{Testing and Commissioning Plan}\label{sec:testing}
\begin{enumerate}
    \item \textbf{Bench testing:} evaluate arena circuitry on a workbench; validate pump dispense accuracy (three-run average within \pm2~ml) and LED/LCD behaviours.
    \item \textbf{Robot subsystem checks:}
    \begin{itemize}
        \item ALFR: confirm line tracking over varying surface reflectivity, verify stop-and-wait logic at each gate.
        \item SARM: verify joystick responsiveness, arm precision, and obstacle relocation repeatability.
    \end{itemize}
    \item \textbf{Integrated dry runs:} simulate the full challenge sequence in Unity, then replicate on hardware with safety observers.
    \item \textbf{Performance logging:} capture ROS~2 bag files during official practice runs; analyze time to completion, pump accuracy, and load cell metrics.
\end{enumerate}

\section{Scoring Readiness Matrix}
\begin{tabularx}{\textwidth}{>{\bfseries}lYY}
\toprule
Rulebook Item & Evidence in this Design & Status \\
\midrule
S.No 1 Arena Circuit & Section~\ref{sec:arena} wiring, power, safety & Ready \\
S.No 2 SARM Omni Robot & Section~\ref{sec:sarm} mobility, control & Ready \\
S.No 3 ALFR & Section~\ref{sec:alfr} sensing and control & Ready \\
S.No 4/5/6 Obstacle Stages & Sections~\ref{sec:alfr} and \ref{sec:sarm} clearance workflow & Ready \\
S.No 7 Peristaltic Pump & Section~\ref{sec:pump} calibration and safety & Ready \\
S.No 8 Custom SARM Design & Section~\ref{sec:sarm} custom CAD & Ready \\
S.No 9 Unique Documentation and Custom ALFR & Sections~\ref{sec:alfr} and \ref{sec:custom} & Ready \\
Time Bonus & Section~\ref{sec:testing} logging for time optimization & Planned \\
\bottomrule
\end{tabularx}

\section{Implementation Roadmap}
\begin{enumerate}
    \item Finalize CAD and fabrication: produce CNC and 3D-print files for SARM chassis and arm; assemble ALFR chassis with sensor mounts.
    \item Assemble arena control box: wire sensors, pump, LEDs, load cell, and LCD; conduct insulation resistance checks.
    \item Deploy ROS~2 stack: configure nodes, parameters, and launch files for both simulation and hardware targets.
    \item Iterate testing: alternate between Unity simulation (parameter tuning) and hardware trials (real-world validation).
    \item Finalize documentation: insert photos, wiring diagrams, parameter tables, and measured calibration data.
\end{enumerate}

\section{Environment Setup and Runbook}
\subsection{Prerequisites}
\begin{itemize}
    \item Operating systems: Ubuntu~22.04 LTS (ROS~2 Humble) for ROS stack; Windows~11 or Ubuntu~22.04 for Unity development.
    \item Software versions: Unity~2022.3 LTS with Burstable compile; ROS~2 Humble; Python~3.10; .NET Framework installed via Unity Hub.
    \item Hardware: Raspberry Pi~4 or Pi~5 for arena control; NVIDIA Jetson optional for SARM autonomy extensions.
\end{itemize}

\subsection{Repository Layout}
\begin{verbatim}
linefollower-lrc-unity/          # Unity project root
  Assets/Scripts/                # Unity ROS integration scripts
  Assets/Scenes/Main.unity       # Primary simulation scene
ROS2/
  lrc_arena_msgs/                # Custom message definitions
  lrc_arena_nodes/               # Arena device ROS 2 nodes & launch
  README.md                      # Build prerequisites & dependency notes
\end{verbatim}

\subsection{Unity Setup Steps}
\begin{enumerate}
    \item Install Unity 2022.3 LTS via Unity Hub with Windows/Linux Build Support.
    \item Clone the repository and open \texttt{linefollower-lrc-unity} in Unity Hub.
    \item Install required packages:
    \begin{itemize}
        \item \texttt{Window \textrightarrow{} Package Manager \textrightarrow{} + \textrightarrow{} Add package from git URL}
        \item \texttt{https://github.com/Unity-Technologies/ROS-TCP-Connector.git?path=/com.unity.robotics.ros-tcp-connector}
        \item \texttt{https://github.com/Unity-Technologies/URDF-Importer.git?path=/com.unity.robotics.urdf-importer}
    \end{itemize}
    \item Generate ROS messages: \texttt{Robotics \textrightarrow{} Generate ROS Messages} and select \texttt{ROS2/lrc\_arena\_msgs}. Confirm generated C\# files under \texttt{Assets/RosMessages}.
    \item Configure ROS settings via \texttt{Robotics \textrightarrow{} ROS Settings}: set ROS IP to the ROS machine address (default 127.0.0.1 for local testing), port 10000, protocol ROS~2.
    \item Prepare the scene: open \texttt{Assets/Scenes/Main.unity}, ensure ALFR prefabs include \texttt{IRArrayPublisher}, \texttt{SonarPublisher}, \texttt{RobotController}; ensure SARM prefab includes \texttt{CmdVelSubscriber}, \texttt{GripperSubscriber}, \texttt{SARcontroller}; confirm Gate objects hold \texttt{GateTriggerPublisher} configured for gates 1--3.
\end{enumerate}

\subsection{ROS 2 Workspace Setup}
\begin{enumerate}
    \item Install ROS~2 Humble and create a workspace:
    \begin{verbatim}
mkdir -p ~/ros2_ws/src
cd ~/ros2_ws
colcon build
source install/setup.bash
    \end{verbatim}
    \item Symlink packages:
    \begin{verbatim}
ln -s /home/astroanax/dev/lrc2025/lrc-final/ROS2/lrc_arena_msgs ~/ros2_ws/src/
ln -s /home/astroanax/dev/lrc2025/lrc-final/ROS2/lrc_arena_nodes ~/ros2_ws/src/
    \end{verbatim}
    \item Install dependencies:
    \begin{verbatim}
cd ~/ros2_ws
rosdep install --from-paths src --ignore-src -r -y
pip install RPi.GPIO hx711 rpi-ws281x RPLCD
    \end{verbatim}
    \item Build packages:
    \begin{verbatim}
colcon build --packages-select lrc_arena_msgs lrc_arena_nodes
source install/setup.bash
    \end{verbatim}
\end{enumerate}

\subsection{Run Instructions (Simulation)}
\begin{itemize}
    \item Terminal 1 --- ROS-TCP endpoint bridge:
\begin{verbatim}
source ~/ros2_ws/install/setup.bash
ros2 run ros_tcp_endpoint default_server_endpoint --ros-args -p ROS_IP:=0.0.0.0
\end{verbatim}
    \item Terminal 2 --- Arena devices launch:
\begin{verbatim}
source ~/ros2_ws/install/setup.bash
ros2 launch lrc_arena_nodes arena_devices.launch.py \
  pump_step_pin:=17 pump_dir_pin:=27 pump_steps_per_ml:=82.0 \
  led_pin:=18 led_count:=30 team_name:="RoboPenguins"
\end{verbatim}
    \item Terminal 3 --- Optional teleoperation:
\begin{verbatim}
source ~/ros2_ws/install/setup.bash
ros2 run teleop_twist_keyboard teleop_twist_keyboard --ros-args -r cmd_vel:=/sarm/cmd_vel
\end{verbatim}
    \item Unity --- press \textbf{Play} once all ROS terminals report ready. Observe ALFR following the line, SARM reacting to teleop commands, and gate triggers firing in the Console.
\end{itemize}

\subsection{Run Instructions (Hardware Integration)}
\begin{enumerate}
    \item Deploy \texttt{lrc\_arena\_nodes} to the Raspberry Pi controlling arena devices; install Python dependencies with \texttt{pip} (use \texttt{--break-system-packages} only if necessary).
    \item Use systemd service files to autostart \texttt{arena\_devices.launch.py} at boot, ensuring the ROS domain ID matches the Unity or desktop configuration.
    \item Connect GPIO wiring per Section~\ref{sec:arena} diagrams.
    \item Calibrate the pump:
    \begin{verbatim}
ros2 param set /pump_controller steps_per_ml 82.5
ros2 topic pub /gates/g1_crossed std_msgs/msg/Bool "data: true"
    \end{verbatim}
    Measure the dispensed volume and update the parameter until within tolerance.
    \item Validate load cell zeroing: tare container, publish \texttt{false} then \texttt{true} to \texttt{/gates/g3\_crossed} to trigger the verification sequence.
\end{enumerate}

\subsection{Troubleshooting Checklist}
\begin{itemize}
    \item \textbf{No ROS connection:} verify ROS IP and port, confirm firewall settings, ensure \texttt{ROS\_DOMAIN\_ID} consistency.
    \item \textbf{Missing C\# message classes:} regenerate Unity messages after building ROS packages.
    \item \textbf{Pump not actuating:} check GPIO permissions (\texttt{sudo usermod -a -G gpio \$USER}), confirm step pin toggles via logic analyzer, validate supply voltage.
    \item \textbf{LED pattern incorrect:} confirm LED count parameter matches the physical strip, ensure level shifter outputs approximately 5~V logic.
    \item \textbf{Load cell drift:} perform HX711 calibration steps; ensure shielded cable and proper grounding.
\end{itemize}

\section{Repository Asset Map and Verification}
\begin{tabularx}{\textwidth}{>{\bfseries}lYY}
\toprule
Path & Purpose & Verification Step \\
\midrule
\texttt{linefollower-lrc-unity/Assets/Scripts/IRArrayPublisher.cs} & Unity publisher for IR array & Console logs on Play confirm 20 Hz publishes \\
\texttt{linefollower-lrc-unity/Assets/Scripts/GateTriggerPublisher.cs} & Gate detection triggers & Scene overlays confirm collider coverage \\
\texttt{ROS2/lrc\_arena\_msgs/msg/IRArray.msg} & Custom ROS~2 message & \texttt{ros2 interface show lrc\_arena\_msgs/msg/IRArray} \\
\texttt{ROS2/lrc\_arena\_nodes/lrc\_arena\_nodes/pump\_controller.py} & Pump management & \texttt{ros2 node info /pump\_controller} lists topics \\
\texttt{ROS2/lrc\_arena\_nodes/launch/arena\_devices.launch.py} & Launch orchestration & \texttt{ros2 launch ... --show-args} verifies parameters \\
\texttt{Docs/Design-Documentation.md} & Markdown source document & Insert team media and diagrams before submission \\
\bottomrule
\end{tabularx}

\section{Conclusion}
The proposed design integrates advanced robotics, fluid handling, and arena automation in full compliance with the Lam Research Challenge 2025 rule set. ROS~2 provides a robust backbone for communication and modular development, while Unity enables rapid iteration through accurate simulation of physics and sensor behavior. Our custom SARM platform, precise peristaltic pump, and resilient ALFR control strategy collectively ensure reliable task execution within the 10-minute limit, positioning RoboPenguins for competitive scoring across all evaluated components.

\section*{Bibliography}
\addcontentsline{toc}{section}{Bibliography}
\begin{thebibliography}{9}
\bibitem{ros_unity}
Unity Technologies, ``ROS--Unity Integration Tutorial,'' Unity Robotics Hub, 2025. Available: \url{https://github.com/Unity-Technologies/Unity-Robotics-Hub}.

\bibitem{urdf_importer}
Unity Technologies, ``URDF Importer Tutorial,'' Unity Robotics Hub, 2025. Available: \url{https://github.com/Unity-Technologies/Unity-Robotics-Hub/tree/main/tutorials/urdf_importer}.

\bibitem{pick_and_place}
Unity Technologies, ``Pick and Place Tutorial,'' Unity Robotics Hub, 2025. Available: \url{https://github.com/Unity-Technologies/Unity-Robotics-Hub/tree/main/tutorials/pick_and_place}.

\bibitem{ros2_doc}
Open Robotics, ``ROS 2 Humble Documentation,'' 2025. Available: \url{https://docs.ros.org/en/humble/}.
\end{thebibliography}

\end{document}
